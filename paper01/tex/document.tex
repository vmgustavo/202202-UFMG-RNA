\documentclass[journal]{IEEEtai}

\usepackage[colorlinks,urlcolor=blue,linkcolor=blue,citecolor=blue]{hyperref}

\usepackage{color,array}

\usepackage{graphicx}

%% \jvol{XX}
%% \jnum{XX}
%% \paper{1234567}
%% \pubyear{2020}
%% \publisheddate{xxxx 00, 0000}
%% \currentdate{xxxx 00, 0000}
%% \doiinfo{TQE.2020.Doi Number}

\newtheorem{theorem}{Theorem}
\newtheorem{lemma}{Lemma}
\setcounter{page}{1}
%% \setcounter{secnumdepth}{0}


\begin{document}


\title{An Improved Extreme Learning Machine Based on Principal Components Analysis For Pruning}


\author{Gustavo V. Maia
\thanks{-}}

\markboth{UFMG - 2022/02 - Artificial Neural Networks: Theory and Applications}
{Gustavo V. Maia}

\maketitle

\begin{abstract}
%The abstract should not exceed 250 words.
%This example is 250 words.
    The abstract
\end{abstract}

\begin{IEEEImpStatement}
    The impact statement.
\end{IEEEImpStatement}

\begin{IEEEkeywords}
artificial neural networks
\end{IEEEkeywords}

\section{Introduction}\label{sec:introduction}

\IEEEPARstart{T}{his} is the introduction

\section{First Section}

The first section

\subsection{First Subsection}

The first subsection

\subsection{Second Subsection}

The second subsection

\begin{equation}
\label{eq:equation}
    \!
\end{equation}

\begin{figure}
\centerline{\includegraphics[width=18.5pc]{fig1.png}}
\caption{Magnetization as a function of applied field. Note that ``Fig.'' is abbreviated. There is a period after the figure number, followed by two spaces. It is good practice to explain the significance of the figure in the caption.}
\label{fig:figure}
\end{figure}

\begin{table}
\caption{Units for Magnetic Properties}
\label{tab:table}
\tablefont%
\setlength{\tabcolsep}{3pt}
\begin{tabular*}{21pc}{@{}|p{23pt}|p{81pt}<{\raggedright\hangindent6pt}|p{123pt}<{\raggedright\hangindent6pt}|@{}}
\hline
Symbol& 
Quantity& 
Conversion from Gaussian and \par CGS EMU to SI$^{\mathrm{a}}$ \\
\hline\\[-17pt]
&&\\
$\Phi $& 
Magnetic flux& 
1 Mx $\to  10^{-8}$ Wb $= 10^{-8}$ V$\,\cdot\,$s \\
$B$& 
Magnetic flux density, magnetic induction& 
1 G $\to  10^{-4}$ T $= 10^{-4}$~Wb/m$^{2}$ \\
\hline
\multicolumn{3}{l}{}\\[-5pt]
\multicolumn{3}{@{}p{21pc}@{}}{\hspace*{9pt}Vertical lines are optional in tables. Statements that serve as captions for 
the entire table do not need footnote letters. }\\
\multicolumn{3}{@{}p{21pc}@{}}{\hspace*{9pt}$^{\mathrm{a}}$Gaussian units are the same as cg emu for magnetostatics; Mx 
$=$ maxwell, G $=$ gauss, Oe $=$ oersted; Wb $=$ weber, V $=$ volt, s $=$ 
second, T $=$ tesla, m $=$ meter, A $=$ ampere, J $=$ joule, kg $=$ 
kilogram, H $=$ henry.}
\end{tabular*}
\label{tab:tab1}
\end{table}


\section{Conclusion}

A conclusion section is not required. Although a conclusion may review the main points of the paper, do not replicate the abstract as the conclusion. A conclusion might elaborate on the importance of the work or suggest applications and extensions.

%%%%%%%%%%%%%%%%%%%%%%%%%%%%%%%%%%%%%%%%%%%%%%%%%%%%%%%%%%%%%%%%%%%%%%%%%%%%%%%%%%%%%%%%%%%%%%%%%%%%%%%%%%%%%%%%%%%%%%%%
%%% REFERENCES %%%%%%%%%%%%%%%%%%%%%%%%%%%%%%%%%%%%%%%%%%%%%%%%%%%%%%%%%%%%%%%%%%%%%%%%%%%%%%%%%%%%%%%%%%%%%%%%%%%%%%%%%
%%%%%%%%%%%%%%%%%%%%%%%%%%%%%%%%%%%%%%%%%%%%%%%%%%%%%%%%%%%%%%%%%%%%%%%%%%%%%%%%%%%%%%%%%%%%%%%%%%%%%%%%%%%%%%%%%%%%%%%%

\section*{References}

\end{document}
